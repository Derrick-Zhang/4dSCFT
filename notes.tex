\documentclass[a4paper,11pt]{article}
\usepackage{jheppub} % for details on the use of the package, please see the JINST-author-manual
\usepackage{lineno}
\linenumbers



%\arxivnumber{1234.56789} % if you have one

\title{\boldmath 4d SCFTs and 2d chiral algebras}

% Collaborations

%% [A] If main author
%% \collaboration{\includegraphics[height=17mm]{collabroation-logo}\\[6pt]
%%  XXX collaboration}

%% or
%% [B] If "on behalf of"
%% \collaboration[c]{on behalf of XXX collaboration}


% Authors
% The "\note" macro will give a warning: "Ignoring empty anchor...", you can safely ignore it.

%% [A] simple case: 2 authors, same institution
%% \author[1]{A. Uthor\note{Corresponding author.}}
%% \author{and A. Nother Author}
%% \affiliation{Institution,\\Address, Country}

%% or, e.g.
%% [B] more complex case: 4 authors, 3 institutions, 2 footnotes
%% \author[a,b]{F. Irst,\note{Now at another university}}
%% \author[c]{S. Econd,}
%% \author[a,2]{T. Hird\note{Also at Some University.}}
%% \author[c,2]{and Fourth}
%% \affiliation[a]{Institution_1,\\Address, Country}
%% \affiliation[b]{Institution_2,\\Address, Country}
%% \affiliation[c]{Institution_3,\\Address, Country}

\author{H. Zhang}
\affiliation{Virginia Tech}
%\affiliation{Another University,\\
%different-address, Country}

% E-mail addresses: only for the corresponding author
\emailAdd{hzhang96@vt.edu}

\abstract{Abstract...}



\begin{document}
\maketitle
\flushbottom

\section{Introduction}
We will mainly follow the paper \cite{Beem:2013sza}. The general idea is that we study subsectors of a 4d $\mathcal{N}=2$ SCFT, which has the structure of a 2d chiral algebra. There exists such a map
\begin{equation}
    \chi: \mathrm{4d}~\mathcal{N}=2~\mathrm{SCFT} \rightarrow \mathrm{2d~Chiral~Algebra}.
\end{equation}
To begin with, the superconformal algebra in 4d is given by $\mathfrak{sl}(4|2)$, and we would like to identify a particular two dimensional conformal subalgebra
\begin{equation}
    \mathfrak{sl}(2) \times \widehat{\mathfrak{sl}(2)} \subset \mathfrak{sl}(4|2),
\end{equation}
with the property that the holomorphic factor $\mathfrak{sl}(2)$ commutes with a nilpotent supercharge $\mathbb{Q}$, while the antiholomorphic factor $\widehat{\mathfrak{sl}(2)}$ is $\mathbb{Q}$-exact.

\section{Review of Chiral Algebras in 2d}
Consider a 2d quantum field theory with global symmetry $SL(2, \mathbb{C})$, whose complexification of Lie algebra is $\mathfrak{sl}(2) \times \overline{\mathfrak{sl}(2)}$, generated by $L_0, L_{\pm 1}$ and $\bar{L}_0, \bar{L}_{\pm 1}$ respectively.

We will assume that the space of local operators decomposes into a direct sum of irreducible highest weight representations, which are labeled by the conformal dimensions $h, \bar{h}$.

Chiral symmetry arises when there exists local operators $\mathcal{O}(z, \bar{z})$, such that
\begin{equation}
    \partial_{\bar{z}} \mathcal{O}(z, \bar{z}) = 0 \quad\Rightarrow\quad \mathcal{O}(z, \bar{z}) := \mathcal{O}(z)
\end{equation}
There are infinitely many conserved charges, given by
\begin{equation}
    \mathcal{O}_n := \oint \frac{dz}{2\pi i} z^{n+h-1} \mathcal{O}(z), \quad \mathrm{or} \quad \mathcal{O}(z) = \sum_n \frac{\mathcal{O}_n}{z^{n+h}}.
\end{equation}
For example, the holomorphic energy-momentum tensor $T(z)$ of conformal dimension $(h, \bar{h}) = (2,0)$, satisfies the following OPE
\begin{equation}
    T(z) T(w) \sim \frac{c/2}{(z-w)^4} + \frac{2 T(w)}{(z-w)^2} + \frac{\partial T(w)}{z-w}.
\end{equation}
Its associated conserved charges obey the Virasoro algebra
\begin{equation}
    [L_m, L_n] = (m-n) L_{m+n} + \frac{c}{12} m (m^2-1) \delta_{m+n,0}.
\end{equation}
Another example is the holomorphic current $J^A(z)$ of conformal dimension $(h, \bar{h}) = (1, 0)$, with OPE
\begin{equation}
    J^A(z) J^B(w) \sim \frac{k \delta^{AB}}{(z-w)^2} + \sum_C i f^{ABC} \frac{J^C(w)}{z-w},
\end{equation}
whose charges obey affine Lie algebra at level $k$
\begin{equation}
    [J^A_m, J^B_n] = \sum_C i f^{ABC} J^C_{m+n} + m k \delta^{AB} \delta_{m+n, 0}.
\end{equation}

\section{Failure in higher dimensions}
For general $d$-dimensional theory, if we restrict our operators to lie in a plane $\mathbb{R}^2 \subset \mathbb{R}^d$, and we calculate correlation functions involving only these operators. Then the corresponding observables will transform covariantly under the subalgebra of the $d$-dimensional conformal algebra that leaves the $\mathbb{R}^2$ fixed
\begin{equation}
    \mathfrak{sl}(2) \times \overline{\mathfrak{sl}(2)} \subset \mathfrak{so}(d+2).
\end{equation}
However, it can be argued that the only ``meromorphic" operators on the plane in a higher dimensional theory is the identity operator, and no chiral symmetry algebra can be constructed.

However, this problem can be resolved if we include supersymmetry and consider superconformal field theories.

\section{Translation invariance from cohomology}
In 4d $\mathcal{N}=1$ theories, we can consider \textbf{chiral operators} (not to be confused with the ``meromorphic operators" considered above) such that
\begin{equation}
    \{Q_\alpha, \mathcal{O}(x)] = 0, \quad \alpha = \pm.
\end{equation}
The translation generators are exact with respect to the chiral supercharges
\begin{equation}
    P_{\alpha \dot{\alpha}} = \{Q_\alpha, \bar{Q}_{\dot{\alpha}}\},
\end{equation}
and consequently, due to Jacobi identity, the derivative of a chiral operator is also exact
\begin{equation}
    [P_{\alpha \dot{\alpha}}, \mathcal{O}(x)] = \{Q_\alpha,\mathcal{O}'(x)].
\end{equation}
The chiral supercharges are nilpotent and anticommute, the cohomology classes of chiral operators with respect to the supercharges $Q_\alpha$ are well defined and independent of the insertion point of the operator. Products of chiral operators form a ring at the level of cohomology, called the chiral ring.

\section{Holomorphy from cohomology}
We will find a nilpotent supercharge with the property that its cohomology classes of local operators transform in an $\mathfrak{sl}(2) \times \widehat{\mathfrak{sl}(2)}$ subalgebra of the full superconformal algebra.

We want to identify a two dimensional conformal subalgebra of the four dimensional superconformal algebra
\begin{equation}
    \mathfrak{sl}(2) \times \widehat{\mathfrak{sl}(2)} \subset \mathfrak{sl}(4|2),
\end{equation}
along with a supercharge $\mathbb{Q}$ such that
\begin{itemize}
    \item $\mathbb{Q}$ is nilpotent: $\mathbb{Q}^2 = 0$
    \item $\mathfrak{sl}(2)$ and $\widehat{\mathfrak{sl}(2)}$ act as generators of holomorphic and anti-holomorphic Mobius transformations on a complex plane $\mathbb{C} \subset \mathbb{R}^4$.
    \item the holomorphic generators of $\mathfrak{sl}(2)$ commutes with $\mathbb{Q}$.
    \item the anti-holomorphic generators of $\widehat{\mathfrak{sl}(2)}$ are $\mathbb{Q}$-commutators.
\end{itemize}

In searching for subalgebras of $\mathfrak{sl}(4|2)$ that keep the plane fixed set-wise, we have two candidates
\begin{itemize}
    \item $\mathfrak{sl}(2|1) \times \mathfrak{sl}(2|1)$, which is the symmetry algebra of 2d $\mathcal{N}=2$ SCFT,
    \item $\mathfrak{sl}(2) \times \mathfrak{sl}(2|2)$, which is the symmetry algebra of 2d $\mathcal{N}= (0,4)$ SCFT.
\end{itemize}
It can be argued that only the second option would produce the desired structure.

\section{Details of superconformal algebras}
\subsection{The four-dimensional superconformal algebra}
The spacetime symmetry algebra for $\mathcal{N}=2$ superconformal field theories in four dimensions is the superalgebra $\mathfrak{sl}(4|2)$. The maximal bosonic subalgebra is $\mathfrak{so}(6, \mathbb{C}) \times \mathfrak{sl}(2)_R \times \mathbb{C}^*$.

The $\mathfrak{so}(6, \mathbb{C})$ conformal algebra, in a spinorial basis, is generated by
\begin{itemize}
    \item Lorentz generators $\mathcal{M}_\alpha^{~\beta}, \mathcal{M}^{\dot{\alpha}}_{~\dot{\beta}}$,
    \item Translations $\mathcal{P}_{\gamma \dot{\gamma}}$,
    \item Special conformal transformation $\mathcal{K}^{\dot{\gamma} \gamma}$,
    \item Dilation $\mathcal{H}$,
\end{itemize}
satisfying certain commutation relations, which are listed in Appendix A of \cite{Beem:2013sza}.

The $\mathfrak{sl}(2)_R$ algebra has a Chevalley basis of generators $\mathcal{R}^\pm$ and $\mathcal{R}$, where
\begin{equation}
    [\mathcal{R}^+, \mathcal{R}^-] = 2 \mathcal{R}, \quad [\mathcal{R}, \mathcal{R}^\pm] = \pm \mathcal{R}^\pm
\end{equation}

The generator of the Abelian factor $\mathbb{C}^*$ is denoted by $r$ and is central in the bosonic part of the algebra.

\bigskip
There are sixteen fermionic generators:
\begin{itemize}
    \item eight Poincar\'{e} supercharges $\mathcal{Q}^\mathcal{I}_\alpha, \widetilde{\mathcal{Q}}_{\mathcal{I} \dot{\alpha}}$,
    \item eight conformal supercharges $\mathcal{S}^\alpha_\mathcal{J}, \widetilde{\mathcal{S}}^{\mathcal{J} \dot{\alpha}}$.
\end{itemize}

\subsection{The two-dimensional superconformal algebra}
Now we consider the superalgebra $\mathfrak{sl}(2|2)$. The maximal bosonic subalgebra is $\mathfrak{sl}(2) \times \mathfrak{sl}(2)_R$ with
\begin{itemize}
    \item $\mathfrak{sl}(2)$ generators $\bar{L}_0, \bar{L}_{\pm 1}$,
    \item $\mathfrak{sl}(2)_R$ generators $\mathcal{R}^\pm, \mathcal{R}$.
\end{itemize}
There are also fermionic supercharges
\begin{itemize}
    \item Poincar\'{e} supercharges $\mathcal{Q}^\mathcal{I}, \widetilde{\mathcal{Q}}_\mathcal{J}$,
    \item superconformal charges $\mathcal{S}_\mathcal{J}, \widetilde{S}^\mathcal{I}$.
\end{itemize}
There is also a central element denoted by $\mathcal{Z}$.


\subsection{Embedding the latter into the former}
Recall that in four dimensions, we have Lorentz generators $\mathcal{M}_\alpha^{~\beta}, \mathcal{M}^{\dot{\alpha}}_{~\dot{\beta}}$, where $\alpha, \beta = \pm$ are spinor indices. In embedding the two-dimensional $\mathcal{N}= (0,4)$ superconformal algebra into the four-dimensional $\mathcal{N}=2$ superconformal algebra, we take the fixed two-dimensional subspace to be the one that is fixed pointwise by the rotation generator
\begin{equation}
    \mathcal{M}^\perp := \mathcal{M}_+^{~+} - \mathcal{M}^{\dot{+}}_{~\dot{+}}.
\end{equation}
The generator of rotations acting within the fixed plane is the orthogonal combination,
\begin{equation}
    \mathcal{M} := \mathcal{M}_+^{~+} + \mathcal{M}^{\dot{+}}_{~\dot{+}}.
\end{equation}

More concretely, we are picking out the plane with $x_1 = x_2 = 0$. Introducing complex coordinates $z := x_3 + i x_4, \bar{z} := x_3 - i x_4$, the two-dimensional conformal symmetry generators in $\mathfrak{sl}(2) \times \mathfrak{sl}(2|2)$ can be expressed in terms of the four-dimensional ones as
\begin{equation}
\begin{aligned}
    L_{-1} = \mathcal{P}_{+\dot{+}}, \quad L_{+1} = \mathcal{K}^{\dot{+}+}, \quad2 L_0 = \mathcal{H} + \mathcal{M},\\
    \bar{L}_{-1} = \mathcal{P}_{-\dot{-}}, \quad \bar{L}_{+1} = \mathcal{K}^{\dot{-}-}, \quad 2 \bar{L}_0 = \mathcal{H} - \mathcal{M}.
\end{aligned}
\end{equation}

\noindent \textbf{Check: the commutators.} For example,
\begin{equation}
\begin{aligned}
    [L_{+1}, L_{-1}] &= [\mathcal{K}^{\dot{+}+}, \mathcal{P}_{+\dot{+}}] = \mathcal{H} + \mathcal{M}^{\dot{+}}_{~ \dot{+}} + \mathcal{M}_+^{~+} = \mathcal{H} + \mathcal{M} = 2L_0.\\
    [L_0, L_{+ 1}] &= \frac{1}{2} [\mathcal{H} + \mathcal{M}, \mathcal{K}^{\dot{+}+}] = \frac{1}{2} [\mathcal{H}, \mathcal{K}^{\dot{+}+}] + \frac{1}{2}[\mathcal{M}_+^{~+}, \mathcal{K}^{\dot{+}+}] + \frac{1}{2}[\mathcal{M}^{\dot{+}}_{~\dot{+}}, \mathcal{K}^{\dot{+}+}]\\
    &= - \frac{1}{2} \mathcal{K}^{\dot{+}+} - \frac{1}{2} \mathcal{K}^{\dot{+}+} + \frac{1}{4}\mathcal{K}^{\dot{+}+} - \frac{1}{2} \mathcal{K}^{\dot{+}+} + \frac{1}{4} \mathcal{K}^{\dot{+}+}\\
    &= - \mathcal{K}^{\dot{+}+} = - L_{+1}.
\end{aligned}
\end{equation}

\bigskip
The embedding of fermionic generators is given by
\begin{equation}
    \mathcal{Q}^\mathcal{I} = \mathcal{Q}^\mathcal{I}_-, \quad \widetilde{\mathcal{Q}}_\mathcal{I} = \widetilde{Q}_{\mathcal{I} \dot{-}}, \quad \mathcal{S}_\mathcal{I} = \mathcal{S}_{\mathcal{I}}^-, \quad \widetilde{\mathcal{S}}^\mathcal{I} = \widetilde{\mathcal{S}}^{\mathcal{I} \dot{-}}.
\end{equation}

The embedding of the central element is given by
\begin{equation}
    \mathcal{Z} = r + \mathcal{M}^\perp.
\end{equation}

\subsection{Finding the twisted supercharge $\mathbb{Q}$}
The supercharges defined above obviously commute with $L_0, L_{\pm 1}$. Therefore, to build the desired $\mathbb{Q}$ as linear combination of these supercharges, we only need to require that the anti-holomorphic generators are $\mathbb{Q}$-exact. There are two possible choices
\begin{itemize}
    \item $\mathbb{Q}_1 := \mathcal{Q}^1 + \widetilde{\mathcal{S}}^2, \quad \mathbb{Q}_1^\dagger := \mathcal{S}_1 + \widetilde{\mathcal{Q}}_2$.
    \item $\mathbb{Q}_2 := \mathcal{S}_1 - \widetilde{\mathcal{Q}}_2, \quad \mathbb{Q}_2^\dagger := \mathcal{Q}^1 - \widetilde{\mathcal{S}}^2$.
\end{itemize}
$\mathbb{Q}_1$ and $\mathbb{Q}_2$ give rise to the same $\mathbb{Q}$-exact generators of an anti-holomorphic $\widehat{\mathfrak{sl}(2)}$ algebra
\begin{equation}
\begin{aligned}
    \{\mathbb{Q}_1, \widetilde{\mathcal{Q}}_1\} &= \{\mathcal{Q}^1_-, \widetilde{\mathcal{Q}}_{1 \dot{-}}\} + \{\widetilde{\mathcal{S}}^{2 \dot{-}}, \widetilde{\mathcal{Q}}_{1\dot{-}}\} = \mathcal{P}_{- \dot{-}} + \mathcal{R}^- = \bar{L}_{-1} + \mathcal{R}^- =: \widehat{L}_{-1},\\
    \{\mathbb{Q}_1, \mathcal{S}_2\} &= \{\mathcal{Q}^1_-, \mathcal{S}_2^-\} + \{\widetilde{\mathcal{S}}^{2 \dot{-}}, \mathcal{S}_2^-\} = - \mathcal{R}^+ + \mathcal{K}^{\dot{-}-} = \bar{L}_{+1} - \mathcal{R}^+ =: \widehat{L}_{+1},\\
    \{\mathbb{Q}_1, \mathbb{Q}_1^\dagger\} &= \{\mathcal{Q}^1_-, \mathcal{S}_1^-\} + \{\mathcal{Q}^1_-, \widetilde{\mathcal{Q}}_{2 \dot{-}}\} + \{\widetilde{\mathcal{S}}^{2 \dot{-}}, \mathcal{S}_1^-\} + \{\widetilde{\mathcal{S}}^{2 \dot{-}}, \widetilde{\mathcal{Q}}_{2\dot{-}}\}\\
    &= \frac{1}{2} \mathcal{H} + \mathcal{M}_-^{~-} - \frac{1}{2} r - \mathcal{R} + \frac{1}{2} \mathcal{H} + \mathcal{M}^{\dot{-}}_{~\dot{-}} + \frac{1}{2} r - \mathcal{R}\\
    &= \mathcal{H} + \mathcal{M}_-^{~-} + \mathcal{M}^{\dot{-}}_{~\dot{-}} - 2 \mathcal{R}\\
    &= \mathcal{H} - \mathcal{M}_+^{~+} - \mathcal{M}^{\dot{+}}_{~\dot{+}} - 2 \mathcal{R}\\
    &= 2(\bar{L}_0 - \mathcal{R}) = : \widehat{L}_0,
\end{aligned}
\end{equation}
\begin{equation}
\begin{aligned}
    \{\mathbb{Q}_2, - \mathcal{Q}^2\} &= \{\mathcal{S}_1^-, - \mathcal{Q}^2_-\} + \{\widetilde{\mathcal{Q}}_{2\dot{-}}, \mathcal{Q}^2_-\} = \mathcal{R}^- + \mathcal{P}_{- \dot{-}} = \bar{L}_{-1} + \mathcal{R}^- =:\widehat{L}_{-1},\\
    \{\mathbb{Q}_2, \widetilde{\mathcal{S}}^1\} &= \{\mathcal{S}_1^-, \widetilde{\mathcal{S}}^{1 \dot{-}}\} - \{\widetilde{\mathcal{Q}}_{2\dot{-}}, \widetilde{\mathcal{S}}^{1 \dot{-}}\} = \mathcal{K}^{\dot{-} -} - \mathcal{R}^+ =: \widehat{L}_{+1}\\
    \{\mathbb{Q}_2, \mathbb{Q}_2^\dagger\} &= \{\mathcal{S}_1^-, \mathcal{Q}^1_-\} - \{\mathcal{S}_1^-, \widetilde{\mathcal{S}}^{2 \dot{-}}\} - \{\widetilde{\mathcal{Q}}_{2\dot{-}}, \mathcal{Q}^1_-\} + \{\widetilde{\mathcal{Q}}_{2\dot{-}}, \widetilde{\mathcal{S}}^{2\dot{-}}\}\\
    &= \frac{1}{2} \mathcal{H} + \mathcal{M}_-^{~-} - \frac{1}{2}r - \mathcal{R} + \frac{1}{2}\mathcal{H} + \mathcal{M}^{\dot{-}}_{~\dot{-}} + \frac{1}{2} r - \mathcal{R}\\
    &= \mathcal{H} + \mathcal{M}_-^{~-} + \mathcal{M}^{\dot{-}}_{~\dot{-}} - 2\mathcal{R}\\
    &= \mathcal{H} - \mathcal{M}_+^{~+} - \mathcal{M}^{\dot{+}}_{~\dot{+}} - 2 \mathcal{R}\\
    &= 2(\bar{L}_0 - \mathcal{R}) = : \widehat{L}_0,
\end{aligned}
\end{equation}
where we have used the traceless condition
\begin{equation}
    \mathcal{M}_{-}^{~-} + \mathcal{M}_+^{~+} + \mathcal{M}^{\dot{-}}_{~\dot{-}} + \mathcal{M}^{\dot{+}}_{~\dot{+}} = 0.
\end{equation}
Notice  that $\widehat{\mathfrak{sl}(2)}$ is an $\mathfrak{sl}(2)_R$ twist of $\overline{\mathfrak{sl}(2)}$, and it does not act on the plane by anti-holomorphic conformal transformations.

In addition, the central element $\mathcal{Z}$ is exact with respect to both supercharges,
\begin{equation}
\{\mathbb{Q}_1, \mathbb{Q}_2\} = \{\mathcal{Q}^1_- + \widetilde{\mathcal{S}}^{2\dot{-}}, \mathcal{S}_1^- - \widetilde{\mathcal{Q}}_{2\dot{-}}\} = - \mathcal{Z}.
\end{equation}



\section{The cohomology classes of local operators}
It can be shown that we can build two supercharges $\mathbb{Q}_{1,2}$ satisfying the above conditions. Now we want to study $\mathbb{Q}_i$-cohomology, and we restrict to local operators in four dimensions. In particular, we will derive the conditions under which an operator $\mathcal{O}(x)$ obeys
\begin{equation}
    \{\mathbb{Q}_i, \mathcal{O}(0)] = 0, \quad \mathcal{O}(0) \neq \{\mathbb{Q}_i, \mathcal{O}'(0)].
\end{equation}
In terms of four dimensional quantum numbers, the conditions can be derived to be
\begin{equation}
    \frac{1}{2}(E - (j_1 + j_2)) - R = 0, \quad r + (j_1 - j_2) = 0.
\end{equation}
The first equation means that the $\widehat{L}_0$-eigenvalue is zero, the second meaning that $\mathcal{Z}$-eigenvalue is zero. Since both $\widehat{L}_0$ and $\mathcal{Z}$ are $\mathbb{Q}_i$-exact, we require $\mathcal{O}(0)$ to lie in the zero eigenspace of both $\widehat{L}_0$ and $\mathcal{Z}$.

Operators obeying these conditions will be called \textbf{Schur operators}. Let us further explain the meaning of these quantum numbers:
\begin{itemize}
    \item $E$ is the conformal dimension/eigenvalue of $\mathcal{H}$
    \item $j_1, j_2$ are $\mathfrak{sl}(2)_1$ and $\mathfrak{sl}(2)_2$ Lorentz quantum numbers/eigenvalues of $\mathcal{M}_+^{~+}$ and $\mathcal{M}^{\dot{+}}_{~\dot{+}}$
    \item $R$ is the $\mathfrak{sl}(2)_R$ charge/eigenvalue of $\mathcal{R}$
    \item $r$ is the generator of $\mathbb{C}^*$ and it is central.
\end{itemize}

As we see in the case of ordinary chiral operators in a supersymmetric theory, those chiral operators are annihilated by a chosen supercharge regardless of the insertion point. Things get a bit more complicated in our current case, because translation generators do not commute with $\mathcal{S}_1^-, \widetilde{\mathcal{S}}^{2\dot{-}}$, appeared in the definition of $\mathbb{Q}$. There is no way to define the translation of a Schur operator from the origin to a point outside of the $(z, \bar{z})$ plane so that it continues to represent a $\mathbb{Q}_i$-cohomology class. However, we can define translation within the plane.

We can define \textbf{twisted-translated operators}
\begin{equation}\label{eqn:twisted-translated-op}
    \mathcal{O}(z, \bar{z}) = e^{zL_{-1} + \bar{z} \widehat{L}_{-1}} \mathcal{O}(0) e^{-z L_{-1} - \bar{z} \widehat{L}_{-1}},
\end{equation}
where $\mathcal{O}(0)$ is a Schur operator. Buy construction, these twisted-translated operators are $\mathbb{Q}_i$-closed, and the cohomology class of these operators is well defined and depends on the insertion point holomorphically,
\begin{equation}
    [\mathcal{O}(z, \bar{z})]_{\mathbb{Q}} \quad \Rightarrow \quad \mathcal{O}(z).
\end{equation}

We can also write the operators (\ref{eqn:twisted-translated-op}) in terms of a more standard basis of local operators at the point $(z,\bar{z})$. Consider a spin $k$ representation of $\mathfrak{sl}(2)_R$ as $\mathcal{O}^{\mathcal{I}_1  \mathcal{I}_2 \cdots \mathcal{I}_{2k}}(z, \bar{z})$, it is argued that Schur operators at the origin occupy the highest-weight states of the corresponding spin $k$ representation, i.e., it is given by $\mathcal{O}^{11...1}(0)$. The twisted-translated operator at any other point is defined as
\begin{equation}
    \mathcal{O}(z, \bar{z}) := u_{\mathcal{I}_1}(\bar{z}) \cdots u_{\mathcal{I}_{2k}}(\bar{z}) \mathcal{O}^{\mathcal{I}_1 ... \mathcal{I}_{2k}}(z, \bar{z}), \quad u_{\mathcal{I}}(\bar{z}) := (1, \bar{z}).
\end{equation}

\section{Chiral OPE}
Now let $\mathcal{O}_1(z, \bar{z})$ be a twisted-translated operator, and $\mathcal{O}_2(0)$ be a Schur operator, the general expression of the OPE is given by
\begin{equation}
    \mathcal{O}_1(z, \bar{z}) \mathcal{O}_2(0) = \sum_k \lambda_{12k} \frac{\bar{z}^{R_1 + R_2 - R_k}}{z^{h_1 + h_2 - h_k} \bar{z}^{\bar{h}_1 + \bar{h}_2 - \bar{h}_k}}\mathcal{O}_k(0).
\end{equation}
where $R_k$ is the $R$-charge of the operator $\mathcal{O}_k$. The two dimensional quantum numbers $h, \bar{h}$ can be written in terms of the four dimensional quantum numbers as
\begin{equation}
    h = \frac{E + (j_1 + j_2)}{2}, \quad \bar{h} = \frac{E - (j_1 + j_2)}{2}.
\end{equation}
Notice that we don't have any constraints on $\mathcal{O}_k$'s on the right hand side for now. Since the LHS is $\mathbb{Q}$-closed, so is the RHS, and we want to decompose it into two groups. One corresponds to Schur operators, and the other corresponds to $\mathbb{Q}$-commutators. One important properties of Schur operators is that $\bar{h} = R$. Therefore, we have
\begin{equation}
    \mathcal{O}_1(z, \bar{z}) \mathcal{O}_2(0) = \sum_k \lambda_{12k} \frac{\mathcal{O}_k(0)}{z^{h_1 + h_2 - h_k}} + \{ \mathbb{Q}_i, \cdots ]
\end{equation}

\section{Short representations of 4d $\mathcal{N}=2$ superconformal algebra}
The unitarity bounds for a superconformal primary operator are given by
\begin{equation}
\begin{aligned}
    E &\geq E_i, & j_i \neq  0,\\
    E &= E_i - 2 \quad \mathrm{or} \quad E \geq E_i & j_i = 0,
\end{aligned}
\end{equation}
where we have defined
\begin{equation}
    E_1 = 2 + 2 j_1 + 2R + r, \quad E_2 = 2 + 2 j_2 + 2R -r.
\end{equation}
Short representations occur when one or more of these bounds are saturated, and we can take different combinations. The supermultiplets that contain Schur operators are denoted by
\begin{itemize}
    \item $\hat{\mathcal{B}}_R$, where the quantum number relations are $j_1 = j_2 = r = 0$, and $E = 2R$,
    \item $\mathcal{D}_{R(0, j_2)}$, where we have $j_1 = 0, r = j_2 + 1$, and $E = 1 + 2 R + j_2$,
    \item $\bar{\mathcal{D}}_{R(j_1, 0)}$, where $j_2 = 0, -r = j_1 + 1$, and $E = 1 + 2 R + j_1$,
    \item $\hat{\mathcal{C}}_{R(j_1, j_2)}$, where $r = j_2 - j_1$, and $E = 2 + 2R + j_1 + j_2$.
\end{itemize}

Recall that for Schur operators, we need to have $r = j_2 - j_1$ and $E = 2R + j_1 + j_2$. ($h = R + j_1 + j_2$)

\subsection{Schur operators in the supermultiplets}
For each supermultiplet, we denote by $\Psi$ the superconformal primary. Each of the above supermultiplets contains exactly one conformal primary Schur operator $\mathcal{O}_\mathrm{Schur}$, which in general is obtained by the action of some Poincar\'{e} supercharges on $\Psi$. We summarize these in the following table ($Q, \widetilde{Q}$ are the complex scalar fields of a hypermultiplet, and $\lambda_\alpha^\mathcal{I}$ and $\widetilde{\lambda}_{\dot{\alpha}}^\mathcal{I}$ are the left- and right-moving fermions of a vector multiplet)
\begin{table}[htbp]
    \renewcommand{\arraystretch}{1.5}
    \centering
    \begin{tabular}{c|c|c|c|c}
        Multiplet & $\mathcal{O}_\mathrm{Schur}$ & $h$ & $r$ & Lagrangian letters \\
        \hline
        $\hat{\mathcal{B}}_R$ & $\Psi^{11...1}$ & $R$ & $0$ & $Q, \widetilde{Q}$\\
        \hline
        $\mathcal{D}_{R(0,j_2)}$ & $\widetilde{\mathcal{Q}}^1_{\dot{+}} \Psi^{11...1}_{\dot{+}...\dot{+}}$ & $R + j_2 + 1$ & $j_2 + \frac{1}{2}$ & $Q, \widetilde{Q}, \widetilde{\lambda}_{\dot{+}}^1$\\
        \hline
        $\bar{\mathcal{D}}_{R(j_1, 0)}$ & $\mathcal{Q}_+^1 \Psi^{11...1}_{+...+}$ & $R + j_1 + 1$ & $-j_1 - \frac{1}{2}$ & $Q, \widetilde{Q}, \lambda_+^1$\\
        \hline
        $\hat{\mathcal{C}}_{R(j_1, j_2)}$ & $\mathcal{Q}_+^1 \widetilde{\mathcal{Q}}^1_{\dot{+}} \Psi^{11...1}_{+...+ \dot{+}...\dot{+}}$ & $R + j_1 + j_2 + 2$ & $j_2 - j_1$
    \end{tabular}
    %\caption{}
    \label{tab:multiplet-schur}
\end{table}

From the face of it, it seems like some of the $\mathcal{O}_\mathrm{Schur}$ do not satisfy the conditions of Schur operators. But bear in mind that $R, j_1, j_2$ are quantum numbers of the left-most multiplet. For example, consider the $\widetilde{Q}_{\dot{+}}^1 \Psi^{11...1}_{\dot{+}...\dot{+}}$ in $\mathcal{D}_{R(0,j_2)}$, the quantum number of the former is given by
\begin{equation}
    j_2' = j_2 + \frac{1}{2}, \quad R' = R + \frac{1}{2}, \quad j_1' = 0,
\end{equation}
which leads to
\begin{equation}
    h = R' + j_1' + j_2' = R + j_2 + 1, \quad r = j_2' - j_1' = j_2 + \frac{1}{2}.
\end{equation}

\subsection{Hall-Littlewood chiral rings}
The Schur operators of type $\hat{\mathcal{B}}_R, \mathcal{D}_{R(0,j_2)}$ and $\bar{\mathcal{D}_R(j_1, 0)}$ can be understood as special cases of conventional $\mathcal{N}=1$ chiral or anti-chiral operators.

Let us focus on the $\mathcal{N}=1$ Poincar\'{e} subalgebra that contains the supercharges
\begin{equation}
    \mathcal{Q}_\alpha^2, \quad \widetilde{\mathcal{Q}}_{2 \dot{\alpha}}.
\end{equation}
Then we ask what subset of Schur operators are also elements of the chiral ring for this $\mathcal{N}=1$ subalgebra. Such operators will be annihilated by both spinorial components of the anti-chiral supercharge $\widetilde{\mathcal{Q}}_{2 \dot{\alpha}}$, $\dot{\alpha} = \dot{\pm}$. These operators have $j_2 = 0$, which are Schur operators of type $\hat{\mathcal{B}}_R$ and $\bar{\mathcal{D}}_{R(j_1,0)}$. These are precisely the operators that contribute to the Hall-Littlewood limit of the superconformal index, and we refer to them as \textbf{Hall-Littlewood operators}. They form a ring called the \textbf{Hall-Littlewood chiral ring}, which is a consistent truncation of the full $\mathcal{N}=1$ chiral ring.

In complete analogy, we may also define a Hall-Littlewood anti-chiral ring, which contains the Schur operators of type $\hat{\mathcal{B}}_R$ and $\mathcal{D}_{R(0,j_2)}$.

Schur operators of type $\hat{\mathcal{B}}_R$ belong to both HL rings. These are half-BPS operators that are annihilated by both $\mathcal{Q}_\alpha^1$ and $\widetilde{\mathcal{Q}}_{2\dot{\alpha}}$. They form a further truncation of the $\mathcal{N} = 1$ chiral ring to the \textbf{Higgs chiral ring}, which gets its name because the vacuum expectation values of these operators parametrize the Higgs branch of the theory.

\subsection{Interpretation of certain multiplets}
Let us look in greater detail at some Schur-type shortened multiplets of particular physical interest:
\begin{itemize}
    \item $\hat{\mathcal{C}}_{0(0,0)}$: Stress-tensor multiplet. The superconformal primary is a scalar operator of dimension two that is a singlet under the $SU(2)_R \times U(1)_r$. The Schur operator is the highest weight component of the $SU(2)_R$ current: $J^{11}_{+\dot{+}}$ of the $SU(2)_R$.
    \item $\hat{\mathcal{C}}_{0(j_1,j_2)}$: Higher-spin currents multiplets. These generalize the stress-tensor multiplet and contain conserved currents of spin higher than two.
    \item $\hat{\mathcal{B}}_{\frac{1}{2}}$: This is the superconformal multiplet of free hypermultiplets.
    \item $\hat{\mathcal{B}}_1$: Flavor-current multiplet
    \item $\mathcal{D}_{0(0,0)} \oplus \bar{\mathcal{D}}_{0(0,0)}$: This is the superconformal multiplet of free $\mathcal{N}=2$ vector multiplets.
\end{itemize}

\section{Symmetry enhancement}
We consider the following map
\begin{equation}
    \chi: \mathrm{4d~SCFT~}\mathcal{T} \to \mathrm{2d~Chiral~ Algebra}
\end{equation}
The 2d chiral algebra $\chi[\mathcal{T}]$ has global $\mathfrak{sl}(2)$ conformal symmetry and possibly other flavor symmetries. There would be enhancement of these global symmetries.

\subsection{Virasoro enhancement of $\mathfrak{sl}(2)$}
The holomorphic $\mathfrak{sl}(2)$ algebra generated by $L_0, L_{\pm 1}$ is enhanced to the full Virasoro algebra, generated by $L_n, n \in \mathbb{Z}$. There is a certain Schur-type shortened stress-tensor multiplet in every theory. We can calculate the OPE of the holomorphic stress-tensors and obtain the central charge
\begin{equation}
    c_{\mathrm{2d}} = -12 c_{\mathrm{4d}}.
\end{equation}
This means that if the 4d theory is unitary, the chiral algebra would be non-unitary.

\bigskip
\noindent\textbf{Explicit calculation.} The Schur operator is the component $J^{11}_{+\dot{+}}$ of the $SU(2)_R$ current $J^{\mathcal{I} \mathcal{J}}_{\alpha \dot{\alpha}}$. The corresponding twisted-translated operator is defined as follows
\begin{equation}
    \mathcal{J}_R(z, \bar{z}) := u_\mathcal{I}(\bar{z}) u_\mathcal{J}(\bar{z}) J^{\mathcal{IJ}}_{+\dot{+}}(z, \bar{z}).
\end{equation}
The OPE of two $SU(2)_R$ currents
\begin{equation}
    J_\mu^\mathcal{IJ}(x) J_\nu^\mathcal{KL}(0) \sim \frac{3 c_\mathrm{4d}}{4 \pi^4} \epsilon^{\mathcal{K}(\mathcal{I}} \epsilon^{\mathcal{J})\mathcal{L}} \frac{x^2 g_{\mu \nu} - 2 x_\mu x_\nu}{x^8} + \frac{2i}{\pi^2} \frac{x_\mu x_\nu x \cdot J^{(\mathcal{K} ( \mathcal{I}} \epsilon^{\mathcal{J}) \mathcal{L}) } }{x^6} + \cdots.
\end{equation}
Then the OPE of twisted-translated Schur operators is given by
\begin{equation}
\begin{aligned}
    \mathcal{J}_R(z, \bar{z}) \mathcal{J}_R(0,0) \sim 
\end{aligned}
\end{equation}

\subsection{Affine enhancement of the flavor symmetry}
On the other hand, the flavor symmetries of $\mathcal{T}$ are always enhanced into affine symmetries of $\chi[\mathcal{T}]$, and the central charges of these flavor symmetries satisfy
\begin{equation}
    k_\mathrm{2d} = -\frac{1}{2} k_\mathrm{4d}.
\end{equation}

\section{Examples of free theories}
\subsection{VOAs of free theories}
\textbf{Free hypermultiplet.}
Let us consider a theory of a single free hypermultiplet. The hypermultiplet itself lies in the short supermultiplet $\hat{\mathcal{B}}_{\frac{1}{2}}$, in which the primary Schur operators are the scalars $Q$ and $\widetilde{Q}$. These are the highest weight states in a pair of $SU(2)_R$ doublets,
\begin{equation}
    Q^\mathcal{I} = \begin{pmatrix}
        Q\\ \widetilde{Q}^*
    \end{pmatrix}, \quad \widetilde{Q}^\mathcal{I} = \begin{pmatrix}
        \widetilde{Q}\\ - Q^*
    \end{pmatrix}.
\end{equation}
The single free hypermultiplet enjoys an $SU(2)_F$ flavor symmetry, under which $Q^\mathcal{I}$ and $\widetilde{Q}^\mathcal{I}$ transform as a doublet. We introduce the following tensor
\begin{equation}
    Q^\mathcal{I}_{\hat{\mathcal{I}}} := \begin{pmatrix}
        Q & \widetilde{Q}\\
        \widetilde{Q}^* & - Q^*
    \end{pmatrix},
\end{equation}
where $\hat{\mathcal{I}}$ is the $SU(2)_F$ index.

The Schur operators in this free theory are all the words that can be constructed out of letters $Q, \widetilde{Q}, \partial_{+\dot{+}}$.
The twisted-translated operators are defined as
\begin{equation}
    Q_{\hat{\mathcal{I}}} (z, \bar{z}) := u_\mathcal{I}(\bar{z}) Q^\mathcal{I}_{\hat{\mathcal{I}}}(z, \bar{z}).
\end{equation}
Let $q_{\hat{\mathcal{I}}} (z) := [Q_{\hat{\mathcal{I}}}(z, \bar{z})]$ be the cohomology class, then we have
\begin{equation}
    q_{\hat{\mathcal{I}}}(z) q_{\hat{\mathcal{J}}}(w) \sim \frac{\epsilon_{\hat{\mathcal{I}} \hat{\mathcal{J}}}}{z-w}.
\end{equation}

Let us unpack these. If we denote the components by $q_{\hat{\mathcal{I}}} = (q, \tilde{q})$, we have
\begin{equation}
    q(z) = [Q(z, \bar{z}) + \bar{z} \widetilde{Q}^*(z, \bar{z})], \quad \tilde{q}(z) = [\widetilde{Q}(z, \bar{z}) - \bar{z} Q^*(z, \bar{z}) ].
\end{equation}
Then
\begin{equation}
\begin{aligned}
    q(z) \tilde{q}(w) &\sim [Q(z, \bar{z}) \widetilde{Q}(w, \bar{w})] - \bar{w} [Q(z, \bar{z}), Q^*(w, \bar{w})] \\
    & \quad + \bar{z} [\widetilde{Q}^*(z, \bar{z})\widetilde{Q}(w, \bar{w})] - \bar{z} \bar{w} [\widetilde{Q}^*(z, \bar{z}) Q^*(w,\bar{w})]\\
    &\sim 0 - \bar{w} \cdot \frac{1}{|z-w|^2} + \bar{z} \cdot \frac{1}{|z-w|^2} - 0\\
    &\sim \frac{1}{z-w}.
\end{aligned}
\end{equation}
The chiral algebra of the free hypermultiplet is thus the free symplectic boson algebra.

****** possible generalization: half hypermultiplets

\bigskip
\noindent \textbf{Free vector multiplet.} Free vectors lie in the short supermultiplet $\bar{\mathcal{D}}_{0(0,0)}$ and $\mathcal{D}_{0(0,0)}$, whose superconformal primaries are the complex scalar $\phi$ and its conjugate $\bar{\phi}$, respectively. The primary Schur operators in these multiplets are the fermions $\lambda_+^1$ and $\tilde{\lambda}_{\dot{1}}^1$, and we can obtain the descendant Schur operators by acting with $\partial_{+\dot{+}}$.

The twisted-translated operators are defined as follows
\begin{equation}
\begin{aligned}
    \lambda(z, \bar{z}) &:= u_\mathcal{I}(\bar{z}) \lambda_+^\mathcal{I}(z, \bar{z}) = \lambda^1_+(z, \bar{z}) + \bar{z} \lambda^2_+(z, \bar{z}),\\
    \tilde{\lambda}(z, \bar{z}) &:= u_\mathcal{I}(\bar{z}) \tilde{\lambda}_{\dot{+}}^\mathcal{I}(z, \bar{z}) = \tilde{\lambda}^1_{\dot{+}}(z, \bar{z}) + \bar{z} \tilde{\lambda}^2_{\dot{+}}(z, \bar{z})
\end{aligned}
\end{equation}
Denote the cohomology class by
\begin{equation}
    \lambda(z) := [\lambda(z, \bar{z})], \quad \tilde{\lambda}(z) := [\tilde{\lambda}(z, \bar{z})]
\end{equation}
Then
\begin{equation}
\begin{aligned}
    \tilde{\lambda}(z) \lambda(w) &\sim [\tilde{\lambda}^1_{\dot{+}}(z, \bar{z}) \lambda^1_+(w, \bar{w})] + \bar{w} [\tilde{\lambda}^1_{\dot{+}}(z, \bar{z}) \lambda_+^2(w, \bar{w})] \\
    & \quad + \bar{z} [\tilde{\lambda}^2_{\dot{+}}(z, \bar{z}) \lambda_+^1(w, \bar{w})] + \bar{z} \bar{w} [\tilde{\lambda}^2_{\dot{+}}(z, \bar{z}) \lambda_+^2(w, \bar{w})]\\
    &\sim 0 + \bar{w} \cdot \frac{-(\bar{z} - \bar{w})}{(z-w)^2 (\bar{z} - \bar{w})^2} + \bar{z} \cdot \frac{(\bar{z} - \bar{w})}{(z-w)^2 (\bar{z}- \bar{w})^2} + 0\\
    &\sim \frac{1}{(z-w)^2}.
\end{aligned}
\end{equation}
Similarly we can calculate other OPEs, and obtain
\begin{equation}
    \tilde{\lambda}(z) \lambda(0) \sim \frac{1}{z^2}, \quad \lambda(z) \tilde{\lambda}(0) \sim - \frac{1}{z^2}.
\end{equation}
We can recognize this chiral algebra as the $(b,c)$ ghost system of weight $(1,0)$,
\begin{equation}
    \tilde{\lambda}(z) := b(z), \quad \lambda(z) := \partial c(z).
\end{equation}
Recall that for $(b,c)$ ghost system, we have $c(z) b(w) \sim 1/(z-w)$.

However, in making this identification, we include $c_0$ mode, which is absent in the chiral algebra generated by $\lambda, \tilde{\lambda}$. To be precise, we should associate this chiral algebra to the \textbf{small algebra} of the $(b,c)$ system. The Fock space of the small algebra is the subspace of the $(b,c)$ Fock space that does not contain $c_0$, or equivalently, the subspace annihilated by $b_0$.

\section{Gauging}

Lagrangian $\mathcal{N}=2$ SCFTs can be described using hypermultiplets and vector multiplets as elementary building blocks. In particular, such an SCFT consists of vector multiplets transforming in the adjoint representation of a semisimple gauge group $G = G_1 \times G_2 \times \cdots \times G_k$, along with a collection of hypermultiplets transforming in some representation of the gauge group such that the one-loop beta functions for all gauge couplings vanish.

The building blocks of the corresponding chiral algebra include
\begin{itemize}
    \item a collection of symplectic bosons
    \item a collection of $(b,c)$-ghost small algebras
\end{itemize}
When the gauge couplings are strictly zero, the chiral algebra is simply obtained by imposing the Gauss law constraint (restricting to the gauge-invariant operators).

We also consider more general SCFTs. Given a general superconformal field theory $\mathcal{T}$ with $G_F$ flavor symmetry, a new SCFT is obtained by gauging a subgroup $G \subset G_F$ provided the gauge coupling beta function vanishes. We will denote the gauged theory with a nonzero gauge coupling $g$ as $\mathcal{T}_G$. By assumption, $\mathcal{T}$ possesses a conserved flavor symmetry current $J^A_{\alpha \dot{\alpha}}$, where $A = 1,..., \dim G$, which is the top component of $\hat{\mathcal{B}}_1$. The gauged theory $\mathcal{T}_G$ is described by minimally coupling an $\mathcal{N}=2$ vector multiplet to $\hat{\mathcal{B}}_1$.

Let us assume that we know $\chi[\mathcal{T}]$. In particular, there will be an affine current $J^A(z)$ at level $k_\mathrm{2d} = - \frac{1}{2}k_\mathrm{4d}$. At zero gauge coupling, the chiral algebra of the gauged theory is obtained by imposing the Gauss law constraint on the tensor product algebra of $\chi[\mathcal{T}]$ with the $(b,c)$ ghost (also imposing the auxiliary condition $b_0^A \psi = 0$ for any state $\psi$).

The affine current associated to the $G$ symmetry in the ghost sector is
\begin{equation}
    J^A_\mathrm{gh} := -i f^{ABC} (c^B b^C).
\end{equation}
The total gauge symmetry current is given by
\begin{equation}
    J^A_\mathrm{tot}(z) := J^A(z) + J^A_\mathrm{gh}(z).
\end{equation}
The Gauss law, or gauge invariance, requires that all physical states should have vanishing total gauge charge, measured by the zero mode of the total gauge symmetry current.

Symbolically, we have the chiral algebra at zero gauge coupling
\begin{equation}
    \chi[\mathcal{T}_G^{(0)}] = \{\psi \in \chi[\mathcal{T}] \otimes (b^A, c^A) \:|\: b_0^A \psi = J^A_{\mathrm{tot}~0} \psi = 0\}.
\end{equation}

But what is the chiral algebra for $\mathcal{T}_G$ with $g \neq 0$?

\section{BRST reduction}
The vanishing of the one-loop beta function amounts to the requirement that in the ungauged theory $\mathcal{T}$, the flavor symmetry central charge is given by
\begin{equation}
    k_\mathrm{4d} = 4 h^\vee,
\end{equation}
which means that in $\chi[\mathcal{T}]$
\begin{equation}
    k_\mathrm{2d} = -2 h^\vee.
\end{equation}
The affine level of the ghost-sector flavor currents $J_\mathrm{gh}$ can be calculated to be $2 h^\vee$. Thus the level of the total affine current $J^A_\mathrm{tot}$ needs to be zero.

In this case, we can construct a nilpotent BRST operator in the chiral algebra. We define
\begin{equation}
    Q_\mathrm{BRST} := \oint \frac{dz}{2\pi i} j_\mathrm{BRST}(z), \quad j_\mathrm{BRST} := c_A \left(J^A + \frac{1}{2} J^A_\mathrm{gh}\right).
\end{equation}
Then the chiral algebra corresponding to the gauged theory at finite coupling is obtained by passing to the cohomology of $Q_\mathrm{BRST}$ relative to the ghost zero modes $b_0^A$,
\begin{equation}
    \chi[\mathcal{T}_G] = \mathcal{H}^*_\mathrm{BRST}[\psi \in \chi[\mathcal{T}] \otimes (b^A, c^A) \:|\: b_0^A \psi = 0].
\end{equation}

The first remark is that we have
\begin{equation}
    \{b_0^A, Q_\mathrm{BRST}\} = J^A_{\mathrm{tot}~0}.
\end{equation}
Therefore, for states in the small algebra ($b_0^A \psi = 0$) that are $Q_\mathrm{BRST}$-closed, they are automatically gauge invariant.

We can rewrite $Q_\mathrm{BRST}$ and separate out the ghost zero modes,
\begin{equation}
    Q_\mathrm{BRST} = c_0^A J^A_{\mathrm{tot}~0} + b_0^A X^A + Q^-,
\end{equation}
where $X^A, Q^-$ have explicit forms in terms of the structure constants and the modes of the ghost fields. One important property of $Q^-$ is that it fails to be nilpotent due to a term proportional to $J^A_{\mathrm{tot}~0}$. However, it is still nilpotent if we restrict ourselves to gauge invariant states. Then we have
\begin{equation}
    \chi[\mathcal{T}_G] = \mathcal{H}^*_{Q^-}[\psi \in \chi[\mathcal{T}] \otimes (\rho^{+A}, \rho^{-A}) \:|\: J^A_{\mathrm{tot}~0} \psi = 0].
\end{equation}




\appendix
\section{Different limits of the superconformal index}



\bibliographystyle{JHEP}
\bibliography{ref}




% Bibliography

%% [A] Recommended: using JHEP.bst file
%% \bibliographystyle{JHEP}
%% \bibliography{biblio.bib}

%% or
%% [B] Manual formatting (see below)
%% (i) We suggest to always provide author, title and journal data or doi:
%% in short all the informations that clearly identify a document.
%% (ii) please avoid comments such as "For a review'', "For some examples",
%% "and references therein" or move them in the text. In general, please leave only references in the bibliography and move all
%% accessory text in footnotes.
%% (iii) Also, please have only one work for each \bibitem.


\end{document}
