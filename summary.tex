\documentclass[a4paper,11pt]{article}
\usepackage{jheppub} % for details on the use of the package, please see the JINST-author-manual
\usepackage{lineno}
\linenumbers



%\arxivnumber{1234.56789} % if you have one

\title{\boldmath 4d SCFTs and 2d chiral algebras}

% Collaborations

%% [A] If main author
%% \collaboration{\includegraphics[height=17mm]{collabroation-logo}\\[6pt]
%%  XXX collaboration}

%% or
%% [B] If "on behalf of"
%% \collaboration[c]{on behalf of XXX collaboration}


% Authors
% The "\note" macro will give a warning: "Ignoring empty anchor...", you can safely ignore it.

%% [A] simple case: 2 authors, same institution
%% \author[1]{A. Uthor\note{Corresponding author.}}
%% \author{and A. Nother Author}
%% \affiliation{Institution,\\Address, Country}

%% or, e.g.
%% [B] more complex case: 4 authors, 3 institutions, 2 footnotes
%% \author[a,b]{F. Irst,\note{Now at another university}}
%% \author[c]{S. Econd,}
%% \author[a,2]{T. Hird\note{Also at Some University.}}
%% \author[c,2]{and Fourth}
%% \affiliation[a]{Institution_1,\\Address, Country}
%% \affiliation[b]{Institution_2,\\Address, Country}
%% \affiliation[c]{Institution_3,\\Address, Country}

\author{H. Zhang}
\affiliation{Virginia Tech}
%\affiliation{Another University,\\
%different-address, Country}

% E-mail addresses: only for the corresponding author
\emailAdd{hzhang96@vt.edu}

\abstract{Abstract...}



\begin{document}
\maketitle
\flushbottom

\section{Introduction}
In \cite{Beem:2013sza}, it was conjectured the following map:
\begin{equation}
    \chi: \mathrm{4d~SCFTs} \quad \rightarrow \quad \mathrm{Chiral~algebras}.
\end{equation}

*** Sending to $\mathbb{Q}$-cohomology

\bigskip
\noindent\textbf{Building Blocks}
\begin{itemize}
    \item For a free hypermultiplet in 4d $\mathcal{N}=2$, the corresponding chiral algebra is the $(\beta, \gamma)$-system (or symplectic bosons)
    \item For a free vector multiplet in 4d $\mathcal{N}=2$, the corresponding chiral algebra is the small $(b,c)$-system.
\end{itemize}

\bigskip
\noindent\textbf{Gauging Prescription}

\newpage
\section{Class-$\mathcal{S}$ theories}

Refer to \cite{Beem:2014rza}




\newpage
\section{4d $\mathcal{N}=4$ SYM}
There is an interesting paper \cite{Buican:2020moo}, which studies two theories:
\begin{itemize}
    \item 4d $\mathcal{N}=4$ $SU(2)$ SYM,
    \item ``$(3,2)$-theory" which is obtained by gauging the diagonal $SU(2)$ of three copies od $D_3(SU(2)) \cong (A_1,A_3)$ Argyres-Douglas theory.
\end{itemize}
There is a relation between the Schur indices of these two theories
\begin{equation}
    \mathcal{I}_\mathrm{SYM} (q = q^3, x = q^{1/2}) = \mathcal{I}_{(3,2)}(q).
\end{equation}
Moreover, they proposed a stronger argument that there is a \textbf{graded vector space isomorphism} between the two VOAs.


This can be generalized to
\begin{itemize}
    \item 4d $\mathcal{N}=4$ SU(N) SYM,
    \item ``$(n,N)$-theory" (See details in \cite{Buican:2020moo}),
\end{itemize}
and
\begin{equation}
    \mathcal{I}_\mathrm{SYM} (q = q^n, x= q^{n/2-1}) = \mathcal{I}_{(n,N)} (q).
\end{equation}


\newpage
\appendix
\section{qDS: general procedure}

\section{Example: qDS on $\widehat{\mathfrak{su}(2)}_k$}
We are not explaining the most general approach. For detailed discussion that can be generalized to arbitrary $\mathcal{W}$-algebra, see section 4.1 of \cite{Beem:2014rza}.

\subsection{Basis}
We choose the following basis of $\mathfrak{su}(2)$,
\begin{equation}
    t_- = \begin{pmatrix}
        0 & 0\\
        1 & 0
    \end{pmatrix}, \quad t_+ = \begin{pmatrix}
        0 & 1\\
        0 & 0
    \end{pmatrix}, \quad t_0 = \frac{1}{2}\begin{pmatrix}
        1 & 0\\
        0 & -1
    \end{pmatrix},
\end{equation}
and we have
\begin{equation}
    [t_+, t_-]=2t_0, \quad [t_0, t_\pm] = \pm t_\pm.
\end{equation}
We can easily read off the structure constants and the Killing form in this basis, which lead to the OPE in the next subsection.

\subsection{BRST cohomology}
The Kac-Moody algebra $\widehat{\mathfrak{su}(2)}_k$ is generated by the currents $J_\pm (z), J_0 (z)$ with OPEs
\begin{align}
    J_0(z) J_\pm (w) &\sim \frac{\pm J_\pm (w)}{z-w},\\
    J_0(z) J_0(w) &\sim \frac{k}{(z-w)^2},\\
    J_+(z) J_-(w) &\sim \frac{2k}{(z-w)^2} + \frac{2J_0(w)}{z-w}.
\end{align}
Now we take the constraint to be $J_-(z) = 1$, and introduce the $(b,c)$-ghost
\begin{equation}
    c(z) b(w) \sim \frac{1}{z-w}.
\end{equation}
Introduce a BRST current
\begin{equation}
    d(z)  = (J_-(z) -1) c(z).
\end{equation}
Then the reduced chiral algebra is defined to be the BRST cohomology of the combined ghost/matter system.

\subsection{Double complex}
Now we further decompose $d(z)$ as $d(z) = d_0(z) + d_1(z)$, where
\begin{equation}
    d_0(z) = - c(z), \quad d_1(z) = J_-(z) c(z),
\end{equation}
and we introduce the following bi-grading
\begin{equation}
\begin{aligned}
    \deg(J_+) &= (1,-1), \quad
    \deg(J_-) = (-1, 1), \quad
    \deg(J_0) = (0,0),\\
    \deg(c) &= (1,0), \quad 
    \deg(b) = (-1, 0).
\end{aligned}
\end{equation}
Then it follows that
\begin{equation}
    \deg(d_0) = (1,0), \quad \deg(d_1) = (0,1).
\end{equation}
The differentials $(d_0, d_1)$ are each differentials in their own right, since
\begin{equation}
    d_0^2 = d_1^2 = d_0 d_1 + d_1 d_0 = 0.
\end{equation}
Therefore, they define a double complex on the Hilbert space of the ghost/matter chiral algebra.


\subsection{Hatted currents}
Before trying to calculate the cohomology of the double complex, it turns out it would be a great simplification if one can also introduce ``hatted" currents.
In our current case, we have
\begin{equation}
    \hat{J}_0 (z) = J_0(z) - (bc)(z),\quad \hat{J}_+(z) = J_+(z).
\end{equation}

Let us denote by $\mathbb{A}_1$ the subalgebra generated by $b(z)$ and $J_-(z)$, and by $\mathbb{A}_2$ the subalgebra generated by $c(z), \hat{J}_0(z), \hat{J}_+(z)$. Moreover, one can check that
\begin{equation}
    d(b(z)) = J_-(z) - 1, \quad d(J_-(z)) = 0.
\end{equation}
It follows that the BRST cohomology of $\mathbb{A}_1$ is trivial, $H^*(\mathbb{A}_1, d) \cong \mathbb{C}$. (\textbf{Why is that?}) For the subalgebra $\mathbb{A}_2$, we have
\begin{equation}\label{eqn:diff-result}
\begin{aligned}
    &d_0(c(z)) = 0, \quad d_1(c(z)) = 0,\\
    &d_0(\hat{J}_0(z)) = c(z), \quad d_1(\hat{J}_0(z)) = 0,\\
    &d_0(\hat{J}_+(z)) = 0, \quad d_1(\hat{J}_+(z)) = 2 k \partial c(z) - 2J_0(z) c(z).
\end{aligned}
\end{equation}
Notice that $d_1(\hat{J}_+(z))$ is not written in terms of the generators of $\mathbb{A}_2$. We should be able to write it in terms of $\hat{J}_0$ instead of $J_0$. So we would have
\begin{equation}
\begin{aligned}
    d_1(\hat{J}_+(z)) &= 2k \partial c(z) - 2\hat{J}_0(z) c(z) -2((bc)c)(z)\\
    &= 2k \partial c(z) - 2\hat{J}_0(z) c(z) + 2 \partial c(z)\\
    &= (2k+2) \partial c(z) - 2 \hat{J}_0(z)c(z),
\end{aligned}
\end{equation}
where we have used the fact that
\begin{equation}
    :(:bc:)c: = -\partial c.
\end{equation}


We have shown that $d(\mathbb{A}_i) \subset \mathbb{A}_i$ ($i=1,2$), and $H^*(\mathbb{A}_1, d) \cong \mathbb{C}$. So by K\"{u}nneth formula,
\begin{equation}
    H^*(\mathbb{A},d) \cong H^*(\mathbb{A}_2, d).
\end{equation}
We only need to calculate the cohomology of $\mathbb{A}_2$.

\subsection{Spectral sequence computation}
To calculate this cohomology, we use the method of a spectral sequence for the double complex $(\mathbb{A}_2, d_0, d_1)$. 

We first compute the cohomology $H^*(\mathbb{A}_2,d_0)$. According to equation \eqref{eqn:diff-result}, $J_+$ is the only generator that is $d_0$-closed, and not $d_0$-exact. Since $c(z)$ is $d_0$-exact, the $d_0$-cohomology is supported entirely at ghost number zero, and the spectral sequence terminates at the first step (\textbf{What does this mean?}) At the level of vector spaces, we have
\begin{equation}
    H^*(\mathbb{A},d) \cong H^*(\mathbb{A}_2, d_0),
\end{equation}
with $J_+$ as the generator.

\bigskip
Next we need to improve this result to produce the VOA structure on this vector space. This is approached by a tic-tac-toe construction. We start from a generator $\psi(z)$ satisfying $d_0(\psi(z)) = 0$, the corresponding chiral algebra generator $\Psi(z)$ is given by
\begin{equation}
    \Psi(z) = \sum_{l}(-1)^l \psi_l(z),
\end{equation}
where
\begin{equation}
    \psi_0(z) := \psi(z), \quad d_1(\psi_l(z)) = d_0(\psi_{l+1}(z)).
\end{equation}

\bigskip

In our current case, we start from $\psi_0(z) = J_+(z)$, now the question is what is $\psi_1(z)$? This part is not explained clearly in \cite{Beem:2014rza}. Another good reference is \cite{deBoer:1993iz}. We start from the most general form. It needs to have bidegree $(0,0)$. Another constraint is from the conformal weight. We need $\psi_l$'s to have the same conformal weight. Also recall that $h(\partial) = 1$. Also define $h(\hat{J}_0) = 1, h(J_+) = 2$. Then the most general form can be written as
\begin{equation}
    \psi_1 = \alpha \hat{J}_0 \hat{J}_0 + \beta \partial \hat{J}_0.
\end{equation}
Now we can calculate 
\begin{equation}
    d_0(\psi_1(z)) = 2\alpha J_0(z) c(z) + (\alpha + \beta) \partial c(z).
\end{equation}
Compare this with
\begin{equation}
    d_1(\psi_0(z)) = 2k\partial c(z) - 2{J}_0(z) c(z).
\end{equation}
We see that
\begin{equation}
    \alpha = -1, \quad \beta = 2k+1.
\end{equation}
Now we also have
\begin{equation}
    d_1(\psi_1(z)) = 0,
\end{equation}
since $d_1 (\hat{J}_0) = 0$. Therefore, the tic-tac-toe construction ends. So we have a generator of the reduced chiral algebra
\begin{equation}
    {\Psi} = \psi_0 - \psi_1 = J_+ +  \hat{J}_0 \hat{J}_0 - (2k+1) \partial \hat{J}_0.
\end{equation}

Before calculating $\Psi(z) \Psi(w)$ OPE, let us calculate 
the OPEs involving $\hat{J}_0$:
\begin{align}
    \hat{J}_0 (z) \hat{J}_0(w) &\sim \frac{k+1}{(z-w)^2},\\
    J_+(z) \hat{J}_0(w) &\sim -\frac{J_+(w)}{z-w}.
\end{align}
Then, it is easy to see that
\begin{equation}
    \Psi(z) \Psi(w) \sim -\frac{2(2+13k +23k^2+12 k^3)}{(z-w)^4} + \frac{4(k+1) \Psi(w)}{(z-w)^2} + \frac{2(k+1) \partial\Psi(w)}{z-w}.
\end{equation}
If we define
\begin{equation}
    \mathcal{J}(z) = \frac{1}{2(k+1)} \Psi(z),
\end{equation}
we obtain
\begin{equation}
    \mathcal{J}(z) \mathcal{J}(w) \sim \frac{c/2}{(z-w)^4} + \frac{2 \mathcal{J}(w)}{(z-w)^2} + \frac{\partial \mathcal{J}(w)}{z-w},
\end{equation}
which is the OPE for the energy-momentum tensor leading to Virasoro algebra, with central charge
\begin{equation}
    c = -\frac{(2+3k)(1+4k)}{k+1} = 13 - \frac{3}{k+1} - 12 (k+1).
\end{equation}
Notice that if we write the level $k = \kappa/2$, we obtain
\begin{equation}
    c = 13 - \frac{6}{\kappa+2} - 6(\kappa+2).
\end{equation}


\section{A pedagogical review of superconformal indices}
None of the following is original. It is mainly based on the presentation by Costis Papageorgakis \cite{Costis}. Another good reference is the lecture by Gadde \cite{Gadde:2020yah}.

\bigskip
\noindent\textbf{Witten index.} In supersymmetric quantum mechanics, the Witten index is given by
\begin{equation}
    \mathcal{I}_\mathrm{Witten} = \mathrm{Tr}_\mathcal{H}\, (-1)^F e^{-\beta H},
\end{equation}
where $\mathcal{H}$ is the Hilbert space of the states, $F$ is the fermion number operator, and $H$ is Hamiltonian, related to the supercharges as $\{Q, \overline{Q}\} = H$. It only receives contributions from SUSY vacua.

Roughly for the superconformal index, we simply replace $H = \{Q, \overline{Q}\}$ by $\{Q, S\}$, where $Q$ is the Poincar\'{e} supercharge, and $S$ is the conformal supercharge. The SUSY vacua corresponds to the short unitary irreducible representations of the superconformal algebra.

\bigskip
\noindent\textbf{Superconformal algebra.}
For $4d \mathcal{N}=2$, the SCA is given by $\mathfrak{su}(2,2|2)$ whose bosonic part is given by
$$
\mathfrak{so}(4,2)  \oplus \mathfrak{su}(2)_R \oplus \mathfrak{u}(1)_r.
$$
The SCA contains the following generators:
Lorentz $M$, $R$-symmetries $R$, dilatations $D$, momentum $P$, special conformal $K$, which are bosonic; and Poincar\'{e} supercharges $Q$, conformal supercharges $S$, which are fermionic.

\bigskip
\noindent\textbf{Representations of SCA.}
The representations are built as follows.
\begin{itemize}
    \item We start with a superconformal primary state $|\psi \rangle$
    $$
    S|\psi\rangle = 0, \quad K|\psi\rangle = 0.
    $$
    Notice that the second equation is the constraint for conformal primaries.
    \item Identify maximal, compact, bosonic subalgebra. In our current case, we have
    $$
    \mathfrak{so}(2) \oplus \mathfrak{so}(4) \oplus \mathfrak{u}(2) \subset \mathfrak{so}(4,2)
    \oplus \mathfrak{u}(2) \subset \mathfrak{su}(2,2|2)    
    $$
    with generators $\Delta, M, R$.
    \item superconformal primaries on in one-to-one correspondence with highest weight states of this compact subalgebra
    $$
    |\psi\rangle \equiv |\Delta; M; R\rangle^\mathrm{h.w.}.
    $$
    \item superconformal descendants are obtained by acting with $Q$'s,
    \item conformal descendants are obtained by acting with $P$'s.
\end{itemize}
Now a basis of the representation space of the SCA is schematically given by
$$
\prod Q^n \prod P^m |\Delta; M; R\rangle^\mathrm{h.w.}
$$
These are generic, long multiplets. Note that there are finite number of superconformal descendants but infinite number of conformal descendants. However, there exists examples where some Poicar\'{e} supercharge annihilates the highest weight state. These are called \textbf{short multiplets}. Short multiplets lead to shortening conditions.

\bigskip
\noindent\textbf{SCI: naive approach.}
Now pick one supercharge $Q$ in the SCA, whose Hermitian conjugate is some conformal supercharge $S$ (in flat space and Euclidean signature), and we have
$$
\{Q, S\} \propto M + \Delta + R.
$$
Now we define
\begin{equation}
    \mathcal{I}:= \mathrm{Tr}_\mathcal{H}\, (-1)^F e^{-\beta \{Q,S\}}.
\end{equation}
Like in Witten index, bosonic/fermionic states for which $\{Q, S\}|\psi\rangle \neq 0$ pairwise cancel. Those with $\{Q, S\}|\psi'\rangle = 0$ contribute and they belong to short multiplets of the SCA with shortening condition $M+ \Delta + R = 0$ (Not all states in short multiplets contribute).

This index is invariant under deformations (short multiplets recombining into long multiplets), and it produces a number.
However, the representations of SCA are infinite-dimensional, so naive index would diverge. One way to resolve this is to refine the index.

\bigskip
\noindent\textbf{Refinement of naive SCI.}
For maximal refinement, we can use the maximal set of SCA generators (or their linear combinations) that commute with $Q$ and $S$ (commutant of SCA). Say the commutant is $c_1, c_2, c_3$, then the refined index will be given by
$$
\mathcal{I}:= \mathrm{Tr}_\mathcal{H}\,(-1)^F e^{-\beta \{Q,S\}} p^{c_1} q^{c_2} t^{c_3},
$$
where $p, q, t$ are called fugacities and are given by exponential of chemical potentials $u$
$$
p = \exp(- \beta u_p), \quad q = \exp(-\beta u_q), \quad t = \exp(-\beta u_t).
$$
Moreover, one can perform additional refinements like using the flavor symmetries.

\bigskip
\noindent\textbf{4d $\mathcal{N}=2$ SCI.}
Now for 4d $\mathcal{N}=2$ SCA, we have a maximal compact bosonic subalgebra
$$
\mathfrak{so}(2) \oplus \mathfrak{su}(2)_1 \oplus \mathfrak{su}(2)_2 \oplus \mathfrak{su}(2)_R \oplus \mathfrak{u}(1)_r,
$$
with quantum numbers $\Delta, j_1, j_2, R, r$. The Poincare supercharges are given by
$$
Q_{I \alpha}, \quad \widetilde{Q}_{I\dot{\alpha}}
$$
The conformal supercharges are given by
$$
S^{I\alpha}, \quad \widetilde{S}^{I \dot{\alpha}}
$$
Now we construct the index, using $\widetilde{Q}_{1\dot{-}}$ and its Hermitian conjugate $\widetilde{S}^{1\dot{-}}$. Then the commutants are given by
\begin{equation}
\begin{aligned}
    \delta_{2+} :=\{Q_{2+}, S^{2+}\} &= \frac{1}{2} D + M_+^+ - R_2^2 \sim \frac{1}{2} \Delta + j_1 - \frac{1}{2}r + R,\\
    \delta_{2-}:=\{Q_{2-}, S^{2-}\} &= \frac{1}{2} D + M_-^- - R^2_2 \sim \frac{1}{2} \Delta - j_1 - \frac{1}{2}r + R,\\
    \tilde{\delta}_{2\dot{+}}:=\{\widetilde{Q}_{2\dot{+}}, \widetilde{S}^{2\dot{+}}\} &= \frac{1}{2} D + M_{\dot{+}}^{\dot{+}} + R^2_2 \sim \frac{1}{2} \Delta + j_2 + \frac{1}{2}r - R,
\end{aligned}
\end{equation}
which are easily verified by considering commutation relations between $\widetilde{Q}_{1\dot{-}}, \widetilde{S}^{1\dot{-}}$, and $D, M, R$.
The shortening condition implies that the quantum number of $\{\widetilde{Q}_{1\dot{-}}, \widetilde{S}^{1\dot{-}}\}$, which is $\frac{1}{2}\Delta - j_2 +\frac{1}{2}r + R$ should be zero. Now we use $\rho, \sigma, \tau$ for fugacities, the superconformal index then is given by
\begin{equation}
    \mathcal{I} = \mathrm{Tr}_\mathcal{H}\, (-1)^F e^{-\beta \{\widetilde{Q}_{1\dot{-}}, \widetilde{S}^{1\dot{-}} \} } \rho^{\delta_1} \sigma^{\delta_2} \tau^{\delta_3}.
\end{equation}

Instead, if we choose $\widetilde{Q}_{2\dot{-}}$, then the commutants are given by
\begin{equation}
\begin{aligned}
    \{Q_{1+}, S^{1+}\} &= \frac{1}{2}D + M_+^+ - R^1_1 \sim \frac{1}{2} \Delta + j_1-\frac{1}{2}r-R =: \delta_{1+}, \\ 
    \{Q_{1-}, S^{1-}\} &= \frac{1}{2}D + M_-^- - R^1_1 \sim \frac{1}{2}\Delta - j_1 - \frac{1}{2}r-R =: \delta_{1-},\\
    \{\widetilde{Q}_{1\dot{+}}, \widetilde{S}^{1\dot{+}}\} &= \frac{1}{2} D + M_{\dot{+}}^{\dot{+}} + R^1_1 \sim \frac{1}{2} \Delta + j_2 +\frac{1}{2} r + R =: \tilde{\delta}_{1\dot{+}}.
\end{aligned}
\end{equation}
The shortening condition implies that the quantum number of $\{\widetilde{Q}_{2\dot{-}}, \widetilde{S}^{2\dot{-}}\}$, which is $\frac{1}{2}\Delta - j_2 +\frac{1}{2}r - R$ should be zero. Now we use $\rho, \sigma, \tau$ for fugacities, the superconformal index then is given by
\begin{equation}
    \mathcal{I} = \mathrm{Tr}_\mathcal{H}\, (-1)^F e^{-\beta \{\widetilde{Q}_{2\dot{-}}, \widetilde{S}^{2\dot{-}}\}} \rho^{\delta_{1+}} \sigma^{\delta_{1-}} \tau^{\tilde{\delta}_{1\dot{+}}}
\end{equation}
Introducing a different set of fugacities $(p,q,t)$ as follows
\begin{equation}
    p = \rho \tau, \quad q = \sigma \tau, \quad t = \tau^2
\end{equation}
so that
\begin{equation}
    \rho = p / t^{1/2}, \quad \sigma = q/t^{1/2}, \quad \tau = t^{1/2}.
\end{equation}
Plug these back into the superconformal index and we obtain
\begin{equation}
\begin{aligned}
    \mathcal{I} &= \mathrm{Tr}_\mathcal{H}\, (-1)^F e^{-\beta \{\widetilde{Q}_{2\dot{-}}, \widetilde{S}^{2\dot{-}}\}} p^{\delta_{1+}} q^{\delta_{1-}} t^{\frac{1}{2}(\tilde{\delta}_{1\dot{+}} - \delta_{1+} - \delta_{1-} )}\\
    &= \mathrm{Tr}_{\mathcal{H}_{\frac{1}{8}\mathrm{BPS}}} (-1)^F p^{\frac{1}{2}\Delta + j_1 - \frac{1}{2}r -R} q^{\frac{1}{2}\Delta - j_1 - \frac{1}{2}r -R} t^{R+r},
\end{aligned}
\end{equation}
which agrees with the SCI defined in \cite{Beem:2013sza}. $\mathcal{H}_{\frac{1}{8}\mathrm{BPS}}$ means the Hilbert space of states that is annihilated by $\widetilde{Q}_{2\dot{-}}$ (the quantum numbers satisfy $\frac{1}{2}\Delta - j_2 + \frac{1}{2}r -R = 0$).

Notice that the indices for different SCAs (but for the same theory) are different. For example, the $\mathcal{N}=4$ index for 4d $\mathcal{N}=4$ SYM is different from the $\mathcal{N}=2$ index.

\bigskip
\noindent\textbf{Practical calculation.}
For Lagrangian theories, the SCI can be calculated as follows
\begin{itemize}
    \item $\mathcal{H}$ comprises the set of all local operators that one can construct from free fields (fields appearing in Lagrangian)
    \item Each field is a word while the composite they build are words
    \item Index over the letters is known as  the single letter index $i(t)$ for $t$ some fugacities.
    \item Equations of motion obeyed by free fields are relations between operators and contribute to the index with opposite sign
    \item The index over all words can then be obtained via the plethystic exponential
    \begin{equation}
        \mathcal{I}(t) = \mathrm{P.E.}[i(t)] = \exp\left[\sum_{n=1}^\infty \frac{1}{n}i(t^n)\right].
    \end{equation}
    For example, if the single letter index $i(t) = t$, then expanding the plethystic exponential, we have
    \begin{equation}
        \mathcal{I}(t) = 1 + t + t^2 + t^3 + \cdots = \frac{1}{1-t}.
    \end{equation}
    If $i(t) = t_1 + t_2$, then we have
    \begin{equation}
        \mathcal{I}(t) = (1 + t_1 + t_1^2 + \cdots) (1+t_2 + t_2^2 + \cdots).
    \end{equation}
    \item Finally, in a gauge theory, we want to consider gauge invariant operators. So we append a group character in the appropriate representation for the single letter index, i.e., $\chi_R(U)$ with $U$ an element of the gauge group  
    \item Characters obey the orthogonality property:
    \begin{equation}
        \int[dU] \chi_R^*(U) \chi_{R'}(U) = \delta_{RR'},
    \end{equation}
    where $[dU]$ is unit normalized Haar measure. We also have
    \begin{equation}
        \chi_{R_1 \otimes R_2} = \chi_{R_1} \cdot \chi_{R_2}.
    \end{equation}
    Therefore,
    \begin{equation}
        \int[dU] \chi_R^*(U) \prod_{i=1}^n \chi_{R_i}(U) = n_R,
    \end{equation}
    where $n_R$ is the number of $R$ representation in $R_1 \otimes R_2 \otimes \cdots R_n$. To keep track of the number of gauge singlets, take $R$ to be the trivial representation, and $\chi_R^*(U) = 1$. Then the final expression for the SCI is
    \begin{equation}
        \mathcal{I} = \int[dU] \mathrm{P.E.}\left[i(t,U)\right] = \int[dU] \exp\left[\sum_{n=1}^\infty \frac{1}{n} i(t^n, U^n)\right].
    \end{equation}
    For example, the ``$\mathcal{N}=2$" index of 4d $\mathcal{N}=4$ SYM (One can think of the fields as a $\mathcal{N}=2$ vector multiplet and a $\mathcal{N}=2$ hypermultiplet in adjoint representation) is given by
    \begin{equation}
        \mathcal{I} = \int[dU] \mathrm{P.E.}\left[(i_V(p,q,t) + i_H(p,q,t))\chi_\mathrm{adj}(U)\right].
    \end{equation}
    In practice, it is hard to perform the gauge integral. We either do low rank, or large $N$, or expand order by order in fugacities.
\end{itemize}

\bigskip
\noindent\textbf{Limits of 4d $\mathcal{N}=2$ SCI.}
We have defined the 4d $\mathcal{N}=2$ SCI as
\begin{equation}
    \mathcal{I} = \mathrm{Tr}_\mathcal{H}\,(-1)^F e^{-\beta \{\widetilde{Q}_{2\dot{-}}, \widetilde{S}^{2\dot{-}} \}} \rho^{\delta_{1+}} \sigma^{\delta_{1-}} \tau^{\tilde{\delta}_{1\dot{+}}}.
\end{equation}
Multiplets that contribute are annihilated by $\widetilde{Q}_{2\dot{-}}$, hence are $1/8$-BPS.
There are interesting fugacity limits one can consider.

\bigskip
\noindent\textbf{MacDonald limit.}
Consider the limit $\rho \to 0$ while $\sigma, \tau$ are fixed, or equivalently $p \to 0$ while $q, t$ fixed. Then only states that are annihilated by both $\widetilde{Q}_{2\dot{-}}$ and $Q_{1+}$ contribute, and 
\begin{equation}
    \mathcal{I} = \mathrm{Tr}_{\mathcal{H}_{\frac{1}{4}\mathrm{BPS}}} (-1)^F q^{\frac{1}{2}\Delta - j_1 - \frac{1}{2}r -R} t^{R+r}.
\end{equation}
The contributing states should satisfy
\begin{equation}
    \frac{1}{2}\Delta - j_2 +\frac{1}{2}r - R, \quad \frac{1}{2} \Delta + j_1 - \frac{1}{2}r- R = 0.
\end{equation}

\bigskip
\noindent\textbf{Schur limit.}
Consider the limit $\sigma = \tau$, we have
\begin{equation}
    \mathcal{I} = \mathrm{Tr}_{\mathcal{H}_{\frac{1}{8}\mathrm{BPS}}}(-1)^F \rho^{\frac{1}{2}\Delta + j_1 - \frac{1}{2}r + R} \sigma^{\Delta - j_1  +j_2}.
\end{equation}
or equivalently the limit $q = t$, we have
\begin{equation}
    \mathcal{I} = \mathrm{Tr}_{\mathcal{H}_{\frac{1}{8}\mathrm{BPS}}} (-1)^F p^{\frac{1}{2} \Delta + j_1 - \frac{1}{2} r  - R} q^{\frac{1}{2}\Delta - j_1 + \frac{1}{2}r}.
\end{equation}
Now we have commutants $\frac{1}{2}D + M_+^+ - R^1_1$, and $\frac{1}{2}D - M_+^+ + \frac{1}{2} R^1_1 + \frac{1}{2}R^2_2$. By construction, these two commute with $\widetilde{Q}_{2\dot{-}}$, but interestingly, they also commute with $Q_{1+}$.
Therefore, we have something like
\begin{equation}
    \mathcal{I} = \mathrm{Tr}\, (-1)^F e^{- u_p \{Q_{1+}, S^{1+}\}} q^{\frac{1}{2}\Delta - j_1 + \frac{1}{2} r},
\end{equation}
which should be independent of $u_p$. Finally we have
\begin{equation}
    \mathcal{I} = \mathrm{Tr}\, (-1)^F q^{\Delta - R}.
\end{equation}


\bigskip
\noindent\textbf{Hall-Littlewood limit.} Consider the limit $p \to 0, q\to 0$, while $t$ is fixed. Then
\begin{equation}
    \mathcal{I} = \mathrm{Tr}_{\mathcal{}}
\end{equation}







\bibliographystyle{JHEP}
\bibliography{ref}




% Bibliography

%% [A] Recommended: using JHEP.bst file
%% \bibliographystyle{JHEP}
%% \bibliography{biblio.bib}

%% or
%% [B] Manual formatting (see below)
%% (i) We suggest to always provide author, title and journal data or doi:
%% in short all the informations that clearly identify a document.
%% (ii) please avoid comments such as "For a review'', "For some examples",
%% "and references therein" or move them in the text. In general, please leave only references in the bibliography and move all
%% accessory text in footnotes.
%% (iii) Also, please have only one work for each \bibitem.


\end{document}
